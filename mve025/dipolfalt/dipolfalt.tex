\documentclass{article}
\usepackage[T1]{fontenc}
\usepackage[utf8]{inputenc}
\usepackage[swedish]{babel}
\usepackage{amsmath}
\usepackage{titlesec}
% \usepackage{lmodern, microtype}

\renewcommand{\vec}[1]{\mathbf{#1}}

\author{Axel Forsman}
\title{Dipolfältet}

\begin{document}
\maketitle

\section*{Problem}
Den elektriska potentialen från en dipol med styrka $\vec{p}$ placerad i Origo ges av
$ \Phi(\vec r) = \frac1{4\pi\epsilon_0} \frac{\vec p \cdot \vec r}{\lvert \vec r \rvert^3} $.

\begin{enumerate}
	\item Utveckla potentialen $\Phi(\vec r)$ i Cartesiska koordinater.
	\item Beräkna den elektriska fältstyrkan $\vec E(\vec r) = -\nabla \Phi(\vec r)$ i Cartesiska koordinater.
	\item Ge ett koordinat-oberoende uttryck för $\vec E(\vec r)$ i termer av
		$\vec p$ och $\vec r$.
	\item Skissera ekvipotentialytorna och de elektriska fältlinjerna.
\end{enumerate}

\section*{Lösning}
\begin{enumerate}
	\item Med Cartesiska koordinater:

		$$ \vec r = \vec r(x, y), \quad \vec p = \vec r(u, v) $$

		Får alltså:

		$$ \Phi(\vec r) = \frac1{4\pi\epsilon_0} \frac{ux + vy}{(\sqrt{x^2 + y^2})^3} $$

	\item

		\begin{equation*}
		\begin{split}
			\vec E(\vec r) &= -\nabla \Phi(\vec r) \\
			&= -\frac1{4\pi\epsilon_0} \left(
			\frac{u (x^2 + y^2)^\frac32 - (ux + vy) \frac32 2x (x^2 + y^2)^\frac12}{(x^2 + y^2)^\frac62} \vec{\hat{i}}
			+ \frac{v (x^2 + y^2)^\frac32 - (ux + vy) \frac32 2y (x^2 + y^2)^\frac12}{(x^2 + y^2)^\frac62} \vec{\hat{j}}
			\right) \\
			&= -\frac1{4\pi\epsilon_0 (x^2 + y^2)^\frac52} \left(
			(u (y^2 - 2x^2) - 3vxy) \vec{\hat{i}}
			+ (v (x^2 - 2y^2) - 3uxy) \vec{\hat{j}}
			\right)
		\end{split}
		\end{equation*}

	\item Med godtyckligt ONHS $(u_1, u_2)$: $\vec r = \vec r(u_1, u_2)$, sätt

		$$ h_i = \left\lvert \frac{\partial \vec r}{\partial \vec u_i} \right\rvert, \quad
		\vec e_i = \frac1{h_i} \frac{\partial \vec r}{\partial u_i}, \quad i = 1, 2 $$

		Har att

		$$ \nabla \phi = \sum_{i=1}^2 \left(\frac1{h_i} \frac{\partial \phi}{\partial u_i}\right) \vec e_i $$

		Får alltså att

		$$ \Phi(\vec r) = $$
\end{enumerate}

\end{document}
